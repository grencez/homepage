
\title{Gentoo Install Notes}
\date{2014.03.05}

\begin{document}

\tableofcontents

\section{Prepare the Disk}

\subsection{Partition Disk}

Start fdisk to work on \ilcode{/dev/sda}
\begin{code}
fdisk /dev/sda
\end{code}

Create boot partition by following:
\begin{code}
new -> primary -> 1st partition 
  -> start at default ->  end 256 MB later
  -> p -> 1 -> -> +256M
\end{code}
make it bootable
\begin{code}
a -> 1
\end{code}

Create swap partition
\begin{code}
n -> p -> 2 -> -> +512M
\end{code}
Set its type to "Linux Swap"
\begin{code}
t -> 2 -> 82
\end{code}

Create root parition taking up the rest
\ilcode{n -> p -> 3 -> ->}
\\finally, write
\ilcode{w}

\subsection{Format the Partitions}

Boot partition as ext3
\begin{code}
mkfs.ext3 /dev/sda1
\end{code}
Create and activate swap
\begin{code}
mkswap /dev/sda2
swapon /dev/sda2
\end{code}
Make \ilcode{sda3} an ext4 filesystem
\begin{code}
mkfs.ext4 /dev/sda3
\end{code}

\subsection{Mount the Partitions}
\begin{code}
mount /dev/sda3 /mnt/gentoo
mkdir /mnt/gentoo/boot
mount /dev/sda1 /mnt/gentoo/boot
\end{code}

\subsection{Get Gentoo System}
Boot up \ilcode{links} and go to \url{http://www.gentoo.org/main/en/mirrors.xml}
choose a mirror, and get stage3 tarball at \ilcode{releases/x86/2007.0/stages/} press \ilcode{d} to download

Do a checksum:
\begin{code}
md5sum -c stage3-i686-2007.0.tar.bz2.DIGESTS
\end{code}

Untar the tarball:
\begin{code}
tar xvjpf stage3-*.tar.bz2 -C /mnt/gentoo
\end{code}
The options mean:
\begin{code}
x - extract, v - verbose, 
j - decompress with bzip2, p - preserve permissions,
f - extract a file
\end{code}
Sometimes you want to leave the verbose part out when using a slow terminal

Do the same with Portage from
\begin{code}
snapshots/portage-latest.tar.bz2
\end{code}
but untar with the command
\begin{code}
tar xvjf portage-latest.tar.bz2 -C /mnt/gentoo/usr
\end{code}

Choose mirrors
\begin{code}
mirrorselect -i -o >> /mnt/gentoo/etc/portage/make.conf
mirrorselect -i -r -o >> /mnt/gentoo/etc/portage/make.conf
\end{code}

\subsection{Copy DNS Info}
\begin{code}
cp -L /etc/resolv.conf /mnt/gentoo/etc/
\end{code}
(the -L option ensures no symbolic link)

\subsection{Mount /proc and /dev}
\begin{code}
mount -t proc none /mnt/gentoo/proc
mount -o bind /dev /mnt/gentoo/dev
\end{code}

\section{Configure the System}

\subsection{Chroot into Gentoo}
change root from \ilcode{/} to \ilcode{/mnt/gentoo}
\begin{code}
chroot /mnt/gentoo /bin/bash
\end{code}
create new environment (variables)
\begin{code}
env-update
\end{code}
load environment variables
\begin{code}
source /etc/profile
export PS1="(chroot) $PS1"
\end{code}

\subsection{Update Portage Tree}
to the latest version
\begin{code}
emerge --sync
\end{code}
for slow console...
\begin{code}
emerge --sync --quiet
\end{code}

\subsection{Kernel Setup}
get kernel
\begin{code}
emerge -q gentoo-sources
\end{code}
and make it (with -s for silent)
\begin{code}
cd /usr/src/linux
make menuconfig
make -s && make -s modules_install
\end{code}
copy kernel image to /boot
\begin{code}
cp arch/x86_64/boot/bzImage /boot/kernel-3.13.5
\end{code}
Use whatever kernel version is appropriate in the name.

\subsection{Specify Modules to Autoload}
List modules for autoloading in \ilcode{/etc/modules.autoload.d/kernel-2.6}
to view all available:
\begin{code}
find /lib/modules/<kernel version>/ \
  -type f -iname '*.o' -or -iname '*.ko'
\end{code}

\subsection{Create fstab}
Syntax:
  <partition> <mount point> <filesystem> <mount options> <needs dump> <fsk>
this example's options
\begin{code}
/dev/sda1    /boot       ext3 defaults,noatime 1 2
/dev/sda2    none        swap sw 0 0
/dev/sda3    /           ext4 noatime 0 1
/dev/cdrom   /mnt/cdrom  auto noauto,usr 0 0
\end{code}

\subsection{Configure the Network}
Name your computer by putting the following line in \ilcode{/etc/conf.d/hostname}.
\begin{code}
HOSTNAME="bat-masterson"
\end{code}

%#domain:
%# /etc/conf.d/net
%# dns_domain_lo="homenetwork"
%
%#To get automatic ip through dhcp, edit
%#  /etc/conf.d/net
%#and add the lines
%#  config_eth0=( "dhcp" )
%#  dhcp_eth0="nodns nontp nonis"
%
%#Start network at Boot
%#  rc-update add net.eth0 default

Define hosts that aren't resolved by the nameserver by editing \ilcode{/etc/hosts}.
\begin{code}
127.0.0.1       localhost       bat-masterson
::1             localhost
\end{code}

\subsection{Get PCMCIA working}
\begin{code}
emerge pcmciautils
\end{code}

\subsection{Set root password}
\begin{code}
passwd
\end{code}

System Information
\begin{code}
/etc/rc.conf
\end{code}


\section{Install Other Stuff}

\subsection{System Logger}
\begin{code}
emerge syslog-ng
rc-update add syslog-ng default
\end{code}

\subsection{Cron Daemon}
Among others, you can use \ilcode{dcron}, \ilcode{fcron}, or \ilcode{vixie-cron} for this, but I chose \ilcode{vixie-cron}.
\begin{code}
emerge vixie-cron
rc-update add vixie-cron default
\end{code}
if dcron or fcron, also do
\begin{code}
crontab /etc/crontab
\end{code}

\subsection{File Indexing}
for the locate tool
\begin{code}
emerge mlocate
\end{code}

\subsection{File System Tools}
possi-ex: xfsprogs, reiserfsprogs, jfsutils
\begin{code}
emerge reiserfsprogs
\end{code}

\subsection{DHCP Client}
\begin{code}
emerge dhcpcd
\end{code}

\subsection{Framebuffer}
find info in \ilcode{/usr/src/linux/Documentation/fb/vesafb.txt}
\begin{code}
emerge vesafb-tng
\end{code}

\subsection{GRUB Setup}
First install Grub.
\begin{code}
emerge grub
\end{code}

Set up grub through grub-install
first setup \ilcode{/etc/mtab}
\begin{code}
grep -v rootfs /proc/mounts > /etc/mtab
\end{code}
With \ilcode{/boot} still mounted, install Grub.
\begin{code}
grub-install /dev/sda
grub-mkconfig -o /boot/grub/grub.cfg
\end{code}
This second command should be rerun any time you install a new kernel.
It reads some settings from \ilcode{/etc/default/grub}.

\section{Finalize the Install}

\subsection{Reboot into new System}
tidy up a little first
\begin{code}
exit
cd
umount /mnt/gentoo/boot /mnt/gentoo/dev /mnt/gentoo/proc /mnt/gentoo
reboot
\end{code}

\subsection{Add User}

\begin{code}
useradd -m grencez
\end{code}

\section{Commands}
Describe the USE flags for a specific package.
\begin{code}
equery --nocolor uses =sys-devel/llvm-3.3-r1 -a
\end{code}

\end{document}

