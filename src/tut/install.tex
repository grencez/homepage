
\title[Install Program as Normal User]{How to Install a Program as a Normal User}
%\author{}
\date{}

\begin{document}

\tableofcontents

If you are a user on a Linux/Unix system without root privileges, you'll probably run into a scenario where you want to install a program that the administrator does not care to install system-wide.
Continue reading to learn how to do such an installation yourself.


\section{Initial Setup}

\subsection{Create Directories}

Run the following command.
\begin{code}
mkdir -p ~/bin ~/local/opt ~/local/src ~/local/stow
\end{code}
For projects with sensible installation steps (e.g., \ilcode{${pkg}}), we can download/build in \ilfile{~/local/src/${pkg}/}, install to \ilfile{~/local/stow/${pkg}/}, then use \ilname{stow} to symbolically link the installed files to their various locations in \ilfile{~/local/bin/}, \ilfile{~/local/lib/}, etc.
In this way, we can remove \ilfile{~/local/src/${pkg}/} without forgetting which files were installed, since they all reside in \ilfile{~/local/stow/${pkg}/}.

If a project \ilcode{${pkg}} does not have a standard install, then we put it in \ilfile{~/local/opt/${pkg}/} and symbolically link to its executable from \ilfile{~/bin/}.
Or simply move the executable to \ilfile{~/bin/}.
This procedure works well enough for me anyway.

\subsection{Set the PATH Environment Variable}
\label{sec:PATH}

The \ilcode{${PATH}} environment variable holds a colon-delimited list of directories where executables should be found.
See what it is now by typing \ilcode{echo ${PATH}}.
We want to make sure that \ilfile{~/bin/} and \ilfile{~/local/bin/} are there.

If you use the \ilname{bash} shell (find out using \ilcode{echo ${SHELL}}), then run the following:
\begin{code}
printf '%s\n' 'export PATH="${HOME}/bin:${HOME}:/local/bin:${PATH}"' >> ~/.bashrc
\end{code}
The changes may not take effect unless you log out and log in again.
To check, remember to \ilcode{echo ${PATH}}.

If you use a \ilname{csh}-based shell like \ilname{tcsh}, then run the following (or preferably switch to something less terrible):
\begin{code}
bash
printf '%s\n' 'setenv PATH "${HOME}/bin:${HOME}/local/bin:${PATH}"' >> "${HOME}/.cshrc"
exit
\end{code}
Note that the other steps of this tutorial require \ilname{bash} or some other POSIX-compliant shell.

\subsection{Install Stow}

The \ilname{stow} (or \ilname{xstow}) program may already be installed, check by running \ilcode{which stow}.
If it is missing, let's install it:

\begin{code}
cd ~/local/src
curl -O http://ftp.gnu.org/gnu/stow/stow-latest.tar.gz
tar xf stow-latest.tar.gz
pkg=$(ls -Avr | grep -m1 -axEe 'stow-[0-9.]+')
cd $pkg

./configure --prefix ~/local/stow/$pkg
make
make install

cd ~/local/stow
./${pkg}/bin/stow $pkg
cd ~/local/src && rm -fr $pkg stow-latest.tar.gz
\end{code}

%pkg=$(ls -AU1 | grep -axEe 'stow-[0-9.]+' | sort -Vr | head -n1)

\section{Normal Procedure}

\subsection{Build/Install}

The idea for build/install is basically what you're used to, but we have to specify an installation path of \ilfile{~/local/stow/$pkg}, rather than the default \ilfile{/usr/local}.

\quicksec{Using Configure Script}
A project that uses a standard \ilcode{./configure && make && make install} process can be installed using the following steps.
\begin{code}
./configure --prefix ~/local/stow/$pkg
make && make install
cd ~/local/stow
stow $pkg
\end{code}
See \href{#sec:eg:configure}{this full example}.

\quicksec{Using CMake}
For projects using \ilname{cmake}, follow this pattern.
\begin{code}
mkdir bld && cd bld
cmake -DCMAKE_INSTALL_PREFIX="${HOME}/local/stow/${pkg}" ..
make && make install
cd ~/local/stow
stow $pkg
\end{code}
See \href{#sec:eg:cmake}{this full example}.

\subsection{Uninstall}

To uninstall a package, we simply invoke the \ilcode{stow -D} command to remove symlinks from \ilfile{~/local/}, then we manually remove the installed files.
It looks like this:
\begin{code}
cd ~/local/stow
stow -D $pkg
rm -fr $pkg
\end{code}

\subsection{Example Using Configure Script}
\label{sec:eg:configure}

Here is an example of installing the \ilname{espresso} logic minimization tool.
\begin{code}
cd ~/local/src
pkg=espresso-ab-1.0
curl -O https://eqntott.googlecode.com/files/${pkg}.tar.gz
tar xf ${pkg}.tar.gz
cd $pkg

./configure --prefix ~/local/stow/$pkg
make
make install

cd ~/local/stow
stow $pkg
cd ~/local/src && rm -fr $pkg ${pkg}.tar.gz
\end{code}

\subsection{Example Using CMake}
\label{sec:eg:cmake}

Here is an example of installing the \ilname{curl} tool that we have used to download files.
\begin{code}
cd ~/local/src
pkg=curl-7.48.0
curl -O https://curl.haxx.se/download/${pkg}.tar.gz
tar xf ${pkg}.tar.gz
cd $pkg

mkdir bld && cd bld
cmake -DCMAKE_INSTALL_PREFIX="${HOME}/local/stow/${pkg}" ..
make -j4
make install

cd ~/local/stow
stow $pkg
cd ~/local/src && rm -fr $pkg ${pkg}.tar.gz
\end{code}

\end{document}

