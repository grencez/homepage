
\title{Generate a Signed SSL Certificate}
\date{}

\begin{document}

\section{What is This?}

If your website doesn't use SSL (HTTPS), then it is bad and wrong.
Without SSL, someone on the network can see exactly what a user is doing on the site.
For security, a trusted third party has to vouch for your public SSL key so that users know they are talking to your site.
This page gives you easy-mode Linux commands to set up SSL.

\section{Initial Configuration}

First grab the client code.
\begin{code}
git clone https://github.com/lukas2511/dehydrated.git
cd dehydrated
\end{code}

Next generate some config files using your \ilcode{$domain}.
\begin{code}
domain=yourdomain.net
echo 'WELLKNOWN=$PWD/public_html/.well-known/acme-challenge' > config.sh
echo "$domain www.$domain" > domains.txt
\end{code}

\section{Generating Keys}

Use \ilname{sshfs} to mount our server's \ilfile{public_html} directory on your local machine.
The following generates SSL keys in \ilfile{certs/$domain/}.
\begin{code}
# Regenererate SSL keys.
rm -fr certs
mkdir -p public_html
sshfs $domain:public_html public_html
mkdir -p public_html/.well-known/acme-challenge
./dehydrated -c -f config.sh
fusermount -u public_html
\end{code}

Now install the keys.
I have to go to \ilname{http://$domain/cpanel}, login, click SSL/TLS, click Install and Manage SSL, select my domain, and then copy/paste the \ilfile{cert.pem} and \ilfile{privkey.pem} files into the Certificate and Private Key fields.
The file contents can quickly be copied using the first two \ilname{xsel} command.
\begin{code}
cd certs/*/
cat cert.pem | xsel -b
cat privkey.pem | xsel -b
echo 'nothing to see here' | xsel -b  # Clear the clipboard selection.
\end{code}

Don't forget to do a \ilcode{git pull origin master} every once in a while to keep the client code updated.

\end{document}

