
\title{Being a Pro with SSH}
%\author{}
\date{}

\begin{document}

\tableofcontents

\section{Reduce your Typing}
\label{sec:sanity}

Let \ilcode{host=apklinkh@guardian.it.mtu.edu}.
We will learn how to log in quickly to this server, even if the name is quite long.

\subsection{Passwordless Login}

We can generate a key pair stored on the current machine at \ilfile{~/.ssh/id\_rsa.pub} (public, safe to share) and \ilfile{~/.ssh/id\_rsa} (private, never share this one).
\begin{code}
ssh-keygen -t rsa -b 4096
\end{code}
Just use the defaults by hitting enter.
Then, we append the content of \ilfile{~/.ssh/id\_rsa.pub} to \ilfile{~/.ssh/authorized\_keys} on the host machine.
\begin{code}
ssh-copy-id $host
\end{code}
And that's all! You should be able to \ilcode{ssh $host} without a password!
For passwordless login to subsequent machines, all you need to do is run the \ilcode{ssh-copy-id} command.

\subsection{Aliases}

If \ilcode{$host} is very long to type, or even if it's short, we can make it shorter by adding the following lines to \ilfile{~/.ssh/config}.
\begin{code}
## ~/.ssh/config
Host wopr
User apklinkh
Hostname guardian.it.mtu.edu
\end{code}
Now we can type \ilcode{ssh wopr} instead of \ilcode{apklinkh@guardian.it.mtu.edu}.

\subsection{Tunneling}
Sometimes you want to \ilcode{ssh} to a host that you can only reach from an internal network.
If you have access to a machine on the internal network, some additions to \ilfile{~/.ssh/config} make it possible to \ilcode{ssh} directly to the desired machine.
In this example, \ilname{superior} is the machine that we want to access, but we must tunnel through \ilname{wopr}.
\begin{code}
## ~/.ssh/config
Host wopr
User apklinkh
Hostname guardian.it.mtu.edu

Host superior
User apklinkh
Hostname superior-login1.research.mtu.edu
ProxyCommand ssh wopr /usr/bin/nc %h %p 2>/dev/null
\end{code}

\end{document}

