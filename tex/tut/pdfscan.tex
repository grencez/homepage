
\title{Fixing up Scanned PDFs}
\date{}

\begin{document}

\tableofcontents

\section{Requirements}

Install the following tools:
\begin{code}
imagemagick
pdfsandwich
scantailor
\end{code}
You probably already have \ilname{imagemagick}.

Say you have a scanned PDF file \ilfile{scan.pdf}.
Move it to a new directory and go to that directory.

\section{Steps}

Assume we have one or more PDFs in a directory \ilfile{/path/to/scan/}.

\subsection{Convert Scanned PDF(s) to Images}

First we must get some images from the scanned PDF.

\quicksec{Option 1: pdfimages}
Try running:
\begin{code}
ls -U | grep -a -e '\.[Pp][Dd][Ff]$' | \
while read -r f
do
  g=$(printf '%s' "$f" | sed -e 's/\.[^.]*$//')
  pdfimages "$f" "$g"
done
\end{code}
The output images are in \ilfile{.pbm} or \ilfile{.ppm} format, depending on if they have color or not.
Take a look at these images, if they are not 1 image per page, then you'll have to do \textbf{option 2}.

The tool we use in the next step does not recognize Netpbm formats, so convert the images to \ilfile{.png} files.
\begin{code}
ls -U | grep -a -e '\.p[bp]m$' | \
while read -r f
do
  g=$(printf '%s' "$f" | sed -e 's/\.[^.]*$//')
  pnmtopng "$f" > "$g".png
done
\end{code}

\quicksec{Option 2: imagemagick}
This option takes a longer time and will not correlate exactly with the original scan.
\begin{code}
ls -U | grep -a -e '\.[Pp][Dd][Ff]$' | \
while read -r f
do
  g=$(printf '%s' "$f" | sed -e 's/\.[^.]*$//')
  convert -density 400 "$f" "$g".png
done
\end{code}

\subsection{Make It Pretty}

Run \ilcode{scantailor} and create a new project in the directory with all of the \ilfile{.png} files.
Let the output directory be \ilfile{out/}.
This tool lets you reorient the pages, split them, make text darker, etc.
It is very helpful for file size and OCR.

After fixing up the formatting and exporting, stitch together the output images into \ilfile{all.pdf}.
\begin{code}
{ ls -v out/*.tif ; echo all.tif ; } | xargs -d '\n' tiffcp
tiff2pdf -o all.pdf -u m -p "letter" -F all.tif
\end{code}
You can also specify a title and author for the PDF using the \ilflag{-t} and \ilflag{-a} flags respectively.

\subsection{Recognize Text}

Here we use the \ilname{pdfsandwich} tool to do OCR.
This runs \ilname{tesseract} in the background using multiple processes if available.
Just run the following command to make a (hopefully) text-searchable \ilfile{final.pdf}.
\begin{code}
pdfsandwich -o final.pdf all.pdf
\end{code}

\end{document}

