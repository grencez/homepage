
\title{Encryption}
\date{}

\begin{document}

\tableofcontents


\section{GnuPG}

Encrypt a secret file using a passphrase.
\begin{code}
gpg -o encrypted.txt.gpg -c secret.txt
# Type passphrase.
\end{code}

Decrypt with a a passphrase.
\begin{code}
gpg -o decrypted.txt -d encrypted.txt.gpg
# Type passphrase.
\end{code}


\section{Cryptsetup}

\subsection{Encrypt Disk Image File with a Passphrase}

Create and encrypt the disk image.
In this case it is 50 megabytes and only uses a 256 bit AES cipher.
It is probably better to use a mix of ciphers.
\begin{code}
head -q -c 50MB /dev/urandom > secret.img
/sbin/cryptsetup -q --cipher aes --key-size 256 luksFormat secret.img
# Need to enter passphrase here.
\end{code}
Then as root (using sudo), format the file system.
\begin{code}
sudo cryptsetup luksOpen secret.img secretimg
# Type passphrase.
sudo mkfs.ext4 /dev/mapper/secretimg
sudo cryptsetup luksClose secretimg
\end{code}

Now mounting to \ilfile{/mnt/point/} will involve:
\begin{code}
sudo cryptsetup luksOpen secret.img secretimg
# Type passphrase.
sudo mount /dev/mapper/secretimg /mnt/point
\end{code}

And unmounting involves:
\begin{code}
sudo umount /mnt/point
sudo cryptsetup luksClose secretimg
\end{code}

% No more need for losetup.
%losetup /dev/loop/0 secret.img
%losetup -d /dev/loop/0 secret.img


\subsection{Encrypt Physical Device with a Key File}

A key file is essentially a longer password that is stored in a file.
For something like an encrypted backup drive, using a key file makes backing up more seamless since no password is required.
It may be reasonable to assume that an attacker will not have access to your own computer, so a key file can even improve security!

The steps are mostly the same, but we must generate a random key and specify it with the \ilflag{--key-file} when invoking \ilcode{cryptsetup}.
In this case, we use the variable \ilcode{$secretdev} to hold \ilfile{/dev/sdc1}, but it could just as well be \ilfile{secret.img} as before.

\begin{code}
secretdev=/dev/sdc1
secretkey=secret.keyfile
mappedname=secretdev
head -q -c 2K /dev/random > $secretkey
sudo cryptsetup -q --cipher aes --key-size 256 --key-file $secretkey luksFormat $secretdev
sudo cryptsetup --key-file $secretkey luksOpen $secretdev $mappedname
sudo mkfs.ext4 /dev/mapper/$mappedname
sudo cryptsetup luksClose $mappedname
\end{code}

That's all there is to it!


%For kernel...
%\begin{code}
%title Gentoo Linux 2.6.30
%root (hd0,0)
%kernel /boot/bzImage-2.6.30 root=/dev/ram0 ramdisk=8192 crypt_root=/dev/sda2
%initrd /boot/initramfs-genkernel-x86_64-2.6.30
%\end{code}

\end{document}

