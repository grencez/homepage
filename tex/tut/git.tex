
\title{Getting by with Git}
%\author{}
\date{}

\begin{document}

\tableofcontents

For this document, assume \ilcode{$proj} is defined as something like:
\begin{code}
proj=my_awesome_project
host=grencez@myserver.net
\end{code}
The following instructions show how to set up a git repository \ilcode{${host}:~/repo/${proj}.git} and use a local copy on your own machine \ilcode{~/code/${proj}}.

\section{Set up a Repository}

\subsection{Global Settings}
It is important for people to know who you are when committing changes.
Create a file \ilfile{~/.gitconfig} (by typing \ilcode{gedit ~/.gitconfig} in a terminal) which contains appropriate lines:
\begin{code}
[user]
  name = Alex Klinkhamer
  email = grencez@youknowthedomain.edu
[push]
  default = matching
\end{code}

\subsection{Making a Repository}

Code shouldn't just be stored on one machine because the hard drive may crash or we may have multiple work machines.
Therefore, we should make a repository on a server that all of our machines can access.
\begin{code}
ssh $host
mkdir -p ~/repo/${proj}.git
cd ~/repo/${proj}.git
\end{code}

\quicksec{Private}
If this is a project just for you, then all we need to do is initialize the repository.
\begin{code}
git init --bare
\end{code}

\quicksec{Public}
If this is a project should be shared with everyone, we must allow people to get to the repository and make it readable and writable by everyone.
\begin{code}
chmod a+x ~/ ~/repo
git init --bare --shared=0666
\end{code}

\quicksec{Shared with Group}
Usually a public repository isn't the greatest idea because anyone on the machine can messs with your code.
If you have the luxury of having a group containing all members, we can restrict access to the group.
Let it be defined as \ilcode{group=my_group}.
\begin{code}
chmod a+x ~/ ~/repo
chgrp $group .
chmod g+rwxs .
git --bare init --shared=group
\end{code}

\section{Basic Usage}
This section describes the minimal amount of knowledge you need to use git.
Note that some commands (\ilcode{git pull} and \ilcode{git push}) assume some default values which were set by \ilcode{git clone}.

\subsection{Check out Code}

Back on your own machine, get a copy of the repository using \ilcode{git clone}.
\begin{code}
mkdir -p ~/code
cd ~/code
git clone ${host}:~/repo/${proj}.git
cd $proj
\end{code}

If you're a member of someone else's project (say, \ilcode{friend=alex}), the \ilcode{clone} command will need to reference their home directory on the server:
\begin{code}
git clone ${host}:~${friend}/repo/${proj}.git
\end{code}

After this initial \ilcode{clone}, it is simple to pull changes other people have made.
\begin{code}
git pull
\end{code}

\subsection{Check in Code}
To add a new file to version control:
\begin{code}
git add $file
\end{code}

To see what files have been added, modified, or are not tracked by git, use the status command.
\begin{code}
git status
\end{code}

To commit all of your changes:
\begin{code}
git commit -a
\end{code}
This will open an editor so you can explain your changes in a \textit{commit message}.
The editor is determined by the \ilcode{$EDITOR} environment variable, which is probably \ilcode{nano} by default... pretty easy to use.
If you only have a short message and don't want to work in an editor, the message may be specified directly.
\begin{code}
git commit -a -m 'Fix size_t comparison warnings'
\end{code}

One can also change the most recent commit or its message (\textbf{ONLY DO THIS IF THE COMMIT HAS NOT BEEN PUSHED}).
\begin{code}
git commit -a --amend
\end{code}

Finally, push your changes to the repository, otherwise nobody will see them!
\begin{code}
git push origin master
\end{code}
If you're not pushing to an empty repository, the following shorthand does the same thing.
\begin{code}
git push
\end{code}

If you still need arguments, you can set the defaults manually with \ilcode{git-config}.
\begin{code}
 git config branch.master.merge refs/heads/master
 git config branch.master.remote origin
\end{code}
To see other options:
\begin{code}
 git config -l
\end{code}
To edit other options:
\begin{code}
 git config -e
\end{code}

\subsection{Misc File Operations}
Add file, remove file, move file, or discard changes.
\begin{code}
git add new-file.c
git rm unwanted-file.c
git mv old-file.c new-file.c
git checkout HEAD file-with-changes.c
\end{code}

If you previously added a file and want to remove it, you must be rather forceful.
\begin{code}
git rm -f --cached unwanted-file.c
\end{code}

See previous commit messages.
\begin{code}
git log
\end{code}

\section{Working with Others}
The above instructions are fine for working by yourself, but what about when others are making changes concurrently?

If \ilcode{git push} complains about your copy not being up to date, you'll need to do a \ilcode{git pull}.
However, there could be conflicts!
(\textbf{TODO}: I don't have enough experience to give advice here.)

If you have uncommitted changes and wish to pull new changes from a teammate, first stash your changes, then have them automatically merged.
(Merges don't always go cleanly though, so copy your changes elsewhere just in case.)
\begin{code}
git stash save
git pull
git stash pop
\end{code}

\section{A Note on Commit Messages}
Most commits do warrant some description.
Imagine if your changes broke something, and someone else (or ``future you'') is tasked with fixing it.
Without a meaningful commit message to read, that person doesn't know your intent in making those original changes, and their change may break something else!
(Side note: Use tests to protect your code from others.)

A commit message should be formatted with the first line being a short description (cut off at 50 characters), followed by an empty line, and then more detailed explanation.
For example, this is one of mine:
\begin{code}
Add normal mapping to raytracer

1. In the raytraced version, one can now specify a normal map in
   object coordinates to give the illusion of a more complex surface.
  a. material.c/h  map_normal_texture()
  a. wavefront-file.c  readin_wavefront()
  a. raytrace.c  fill_pixel()
     - This function is getting too complicated...

2. When determining the normal of the track's surface, be sure the damn
   thing is normalized! This only affects tracks without vertex normals.
  b. motion.c  apply_track_gravity()

3. Clean up some parse code with the addition of a new macro.
  a. util.h  AccepTok()
  c. wavefront-file.c
    + readin_wavefront()
    + readin_materials()
  c. dynamic-setup.c  readin_Track()
\end{code}
This is just my style, but I describe changes in order of priority and hierarchically by: intent (number prefix), file (letter prefix), function or class (`+' prefix, or on same line if there's room), and extra description (`-' prefix).
The letter prefixes signify the type of change: \textit{c} means change code, \textit{b} means bug fix, \textit{a} means add code (new functionality), \textit{r} means remove code, and \textit{d} means comment/documentation changes (if the file also contains code).
This format works pretty well, though the letters may be a bit pedantic (former colleagues used a `*' prefix instead).

\end{document}

